% Macros for writing latex math

\makeatletter

\newcommand{\pto}{}% just for safety
\newcommand{\pgets}{}% just for safety
\DeclareRobustCommand{\pto}{\mathrel{\mathpalette\p@to@gets\to}}
\DeclareRobustCommand{\pgets}{\mathrel{\mathpalette\p@to@gets\gets}}
\newcommand{\p@to@gets}[2]{%
	\ooalign{\hidewidth$\m@th#1\mapstochar\mkern5mu$\hidewidth\cr$\m@th#1\to$\cr}%
}
\makeatother

% Commonly used general concepts
\newcommand{\code}[1]{\texttt{#1}}
\newcommand{\absolute}[1]{\lvert{}{#1}\rvert{}}
\newcommand{\semantics}[1]{\llbracket{}{#1}\rrbracket{}}
\newcommand{\inverse}[1]{{#1}^{-1}}
\newcommand{\dual}[1]{{#1}^{op}}
\newcommand{\kleeneStar}[1]{{#1}^*}
\newcommand{\binprod}[2]{{#1} \times {#2}}
\newcommand{\bincoprod}[2]{{#1} + {#2}}


\newcommand{\rewriteStep}[4]{{#1} \stackrel{{#2}@{#3}}{\rightsquigarrow} {#4}}





% Arrows
\newcommand{\arrow}[3]{{#2}: {#1} \to {#3}}
\newcommand{\inlinearrow}[3]{{#1} \xrightarrow{#2} {#3}}
\newcommand{\longinlinearrow}[3]{\xymatrix{{#1} \ar[r]|{#2} & {#3}}}
\newcommand{\partialarrow}[3]{{#2}: {#1} \rightharpoonup {#3}}
\newcommand{\partialinlinearrow}[3]{{#1} \xrightharpoonup{#2} {#3}}
\newcommand{\partiallonginlinearrow}[3]{\xymatrix{{#1} \ar@{-^{`}}[r]|{#2} & {#3}}}
\newcommand{\relationarrow}[3]{{#2}: {#1} \pto {#3}}
\newcommand{\relationinlinearrow}[3]{{#1} \overset{#2}{\pto} {#3}}
\newcommand{\relationlonginlinearrow}[3]{\xymatrix{{#1} \ar@{-|>}[r]|{#2} & {#3}}}
\newcommand{\inclusionarrow}[3]{{#2}: {#1} \hookrightarrow {#3}}
\newcommand{\inlineinclusionarrow}[3]{{#1} \xhookrightarrow{#2} {#3}}
\newcommand{\longinlininclusionearrow}[3]{\xymatrix{{#1} \ar@{^{(}->}[r]|{#2} & {#3}}}
\newcommand{\monoarrow}[3]{{#2}: {#1} \rightarrowtail {#3}}
\newcommand{\inlinemonoarrow}[3]{{#1} \xrightarrowtail{#2} {#3}}
\newcommand{\longinlinemonoarrow}[3]{\xymatrix{{#1} \ar@{>->}[r]|{#2} & {#3}}}
\newcommand{\epiarrow}[3]{{#2}: {#1} \twoheadrightarrow {#3}}
\newcommand{\inlineepiarrow}[3]{{#1} \xtwoheadrightarrow{#2} {#3}}
\newcommand{\longinlineepiarrow}[3]{\xymatrix{{#1} \ar@{->>}[r]|{#2} & {#3}}}

\newcommand{\worra}[3]{{#2}: {#1} \leftarrow {#3}}
\newcommand{\inlineworra}[3]{{#1} \xleftarrow{#2} {#3}}
\newcommand{\longinlineworra}[3]{\xymatrix{{#1} & {#3} \ar[l]|{#2} }}

\newcommand{\inlinespan}[5]{{#1} \xleftarrow{#2} {#3} \xrightarrow{#4} {#5}}
\newcommand{\inlinecospan}[5]{{#1} \xrightarrow{#2} {#3} \xleftarrow{#4} {#5}}

% Commonly used concepts where notions may have to be tweaked from time to time
\newcommand{\compose}[2]{{#2} \circ {#1}}
\newcommand{\powerset}[1]{\wp{}{#1}}
\newcommand{\identity}[1]{id_{#1}}
\newcommand{\isomorphic}{\cong}
\newcommand{\cat}[1]{\mathbb{#1}}
\newcommand{\functor}[1]{\ensuremath{#1}}
\newcommand{\homset}[3]{{#1}({#2},{#3})}


\newcommand{\catArrows}[1]{\cat{#1}^{\to}}
\newcommand{\catObjects}[1]{\absolute{{#1}}}
\newcommand{\sliceCat}[2]{{#1}\downarrow{}{#2}}
\newcommand{\cosliceCat}[2]{{#1}\uparrow{}{#2}}
\newcommand{\proofStep}[1]{\langle{}\mbox{#1}\rangle{}}


% Adjoints
\newcommand{\leftAdjointTo}[2]{{#1}\dashv{}{#2}}
\newcommand{\unit}[1]{\eta_{#1}}
\newcommand{\counit}[1]{\varepsilon_{#1}}

% Subobject functor
\newcommand{\Sb}[1]{Sub({#1})}
\newcommand{\subob}{\sqsubseteq}
\newcommand{\supob}{\sqsupseteq}
\newcommand{\subobject}[1]{\left[{#1}\right]}
\newcommand{\preimg}[2]{{#1}^{-1}{#2}}
\newcommand{\image}[1]{Im({#1})}


% pullback functor
\newcommand{\pbFunctor}[2]{{#1}^{-1}{#2}}
\newcommand{\upperAdjoint}[2]{\forall_{#1}{#2}}
\newcommand{\lowerAdjoint}[2]{\exists_{#1}{#2}}


% Some functor variables
\newcommand{\F}{\functor{F}}
\newcommand{\G}{\functor{G}}
\newcommand{\U}{\functor{U}}
\renewcommand{\U}{\mathcal{U}}
\newcommand{\Graping}{\Gamma}



% Diagrams

\newcommand{\diagr}[1]{\mathcal{#1}}
\newcommand{\diagrShape}{\cat{S}}


% Partial arrows
\newcommand{\dodef}[1]{dom({#1)}}
\newcommand{\inclPart}[1]{\subseteq_{#1}}
\newcommand{\totalPart}[1]{{#1}}
\newcommand{\inlinespanofpartialMonomap}[5]{
	\xymatrix{{#1} & {#3} \ar@{>->}[l]_{#2}\ar[r]^{#4} & {#5}}
}
\newcommand{\inlinespanofpartialmap}[5]{{#1} \xhookleftarrow{#2} {#3} \xrightarrow{#4} {#5}}
%\newcommand{\inlinespan}[5]{{#1} \xleftarrow{#2} {#3} \xrightarrow{#4} {#5}}
%\newcommand{\inlinecospan}[5]{{#1} \xrightarrow{#2} {#3} \xleftarrow{#4} {#5}}
\newcommand{\partArrName}[2]{[{#1},{#2}\rangle}
%\newcommand{\partialarrow}[3]{{#2}: {#1} \rightharpoonup {#3}}
\newcommand{\subobjClass}[1]{[#1]}
\newcommand{\subobjLessEq}{\sqsubseteq}
\newcommand{\subobjMeet}{\sqcap}
\newcommand{\subobjJoin}{\sqcup}
\newcommand{\subobjBigJoin}{\bigsqcup}

\newcommand{\inlinepartialarrow}[3]{{#1} \xrightharpoonup{#2} {#3}}

% Category of all Sets and functions
\newcommand{\catSet}{\mathbb{S}\mathbb{E}\mathbb{T}}
% Category of all small Categories and morphisms
\newcommand{\catCat}{\mathbb{C}\mathbb{A}\mathbb{T}}
% Category of partial graphs
\newcommand{\relCat}[1]{\code{Rel}(#1)}
\newcommand{\parCat}[1]{\code{Par}(#1)}
\newcommand{\spanCat}[1]{\code{Span}(#1)}
\newcommand{\algCat}[1]{\code{Alg}(#1)}


% Cat of comprehensive systems
\newcommand{\CS}{\cat{C}\cat{S}}
\newcommand{\RCS}{\cat{R}\cat{C}\cat{S}}
\newcommand{\TRCS}[1]{\cat{T}\cat{R}\cat{C}\cat{S}({#1})}
\newcommand{\catBase}{\cat{B}}
\newcommand{\catSchema}{\cat{I}}
\newcommand{\cs}[1]{\mathbf{#1}}
\newcommand{\csLoopVar}{j}
\newcommand{\csLoopVarAll}{i}
\newcommand{\csArrow}[3]{\mathbf{\arrow{#1}{#2}{#3}}}
\newcommand{\csArrowDouble}[3]{\mathbf{\arrow{#1}{#2}{#3}}}
\newcommand{\csWitns}[1]{{#1}^{0}}
\newcommand{\csProj}[2]{\lowercase{#1}^{#2}}
\newcommand{\csProjForall}[1]{\lowercase{#1}^{\csLoopVar}}
\newcommand{\csProjDom}[2]{\dodef{\lowercase{#1}^{#2}}}
\newcommand{\csProjDomForall}[1]{\dodef{\lowercase{#1}^{\csLoopVar}}}
\newcommand{\csProjIncl}[2]{\inclPart{\lowercase{#1}^{#2}}}
\newcommand{\csProjInclForall}[1]{\inclPart{\lowercase{#1}^{\csLoopVar}}}
\newcommand{\csProjTotal}[2]{\totalPart{\lowercase{#1}^{#2}}}
\newcommand{\csProjTotalForall}[1]{\totalPart{\lowercase{#1}^{\csLoopVar}}}
\newcommand{\csCmpnt}[2]{{#1}^{#2}}
\newcommand{\csCmpntForall}[1]{{#1}^{\csLoopVar}}
\newcommand{\csArrOnWtns}[1]{{#1}_0}
\newcommand{\csArrOnCmpnt}[2]{{#1}_{#2}}
\newcommand{\csArrOnCmpntForall}[1]{{#1}_{\csLoopVar}}
\newcommand{\csArrOnDodef}[2]{{#1}_{-{#2}}}
\newcommand{\csArrOnDodefForall}[1]{{#1}_{-{\csLoopVar}}}
%Paper specific
\newcommand{\specialmonos}{\mathscr{M}}
\newcommand{\specialmonoarrow}[3]{\xymatrix{{#1} \ar@{(->}[r]|{#2} & {#3}}}








% All the macros for creating Defitions, Theorems, Proposition etc.
\newtheorem{theorem}{Theorem}
\newtheorem{proposition}{Proposition}
\newtheorem{myfact}{Fact}
\newtheorem{myconj}{Conjecture}
\newtheorem{lemma}{Lemma}
\newtheorem{myclaim}{Claim}
\newtheorem{corollary}{Corollary}
\newtheorem{myparadox}{Paradox}


\theoremstyle{definition}
\newtheorem{definition}{Definition}
\newtheorem{counterexmpl}{Counterexample}
\newtheorem{myaxiom}{Axiom}
\newtheorem{myprinc}{Principle}
\newtheorem{myobs}{Observation}
\newtheorem{myreq}{Requirement}


\theoremstyle{remark}
\newtheorem{myexampl}{Example}
\newtheorem{myremark}{Remark}
\newtheorem{myconv}{Convention}
\newtheorem{myexerc}{Exercise}


\newcommand{\mkTheorem}[3]{
	\begin{mytheo}[{#1}]
	\label{#2}
	{#3}
	\end{mytheo}
}

\newcommand{\mkProposition}[3]{
	\begin{myprop}[{#1}]
		\label{#2}
		{#3}
	\end{myprop}
}

\newcommand{\mkFact}[3]{
	\begin{myfact}[{#1}]
		\label{#2}
		{#3}
	\end{myfact}
}

\newcommand{\mkConjecture}[3]{
	\begin{myconj}[{#1}]
		\label{#2}
		{#3}
	\end{myconj}
}

\newcommand{\mkLemma}[3]{
	\begin{mylemma}[{#1}]
		\label{#2}
		{#3}
	\end{mylemma}
}

\newcommand{\mkClaim}[3]{
	\begin{myclaim}[{#1}]
		\label{#2}
		{#3}
	\end{myclaim}
}
\newcommand{\mkCorollary}[3]{
	\begin{mycorr}[{#1}]
		\label{#2}
		{#3}
	\end{mycorr}
}
\newcommand{\mkParadox}[3]{
	\begin{myparadox}[{#1}]
		\label{#2}
		{#3}
	\end{myparadox}
}


\newcommand{\mkDefinition}[3]{
	\begin{mydef}[{#1}]
	\label{#2}
	{#3}
	\end{mydef}
}

\newcommand{\mkAxiom}[3]{
	\begin{myaxiom}[{#1}]
		\label{#2}
		{#3}
	\end{myaxiom}
}

\newcommand{\mkPrinciple}[3]{
	\begin{myprinc}[{#1}]
		\label{#2}
		{#3}
	\end{myprinc}
}

\newcommand{\mkObservation}[3]{
	\begin{myobs}[{#1}]
		\label{#2}
		{#3}
	\end{myobs}
}

\newcommand{\mkRequirement}[3]{
	\begin{myreq}[{#1}]
		\label{#2}
		{#3}
	\end{myreq}
}

\newcommand{\mkExample}[3]{
	\begin{myexampl}[{#1}]
		\label{#2}
		{#3}
	\end{myexampl}
}

\newcommand{\mkRemark}[3]{
	\begin{myremark}[{#1}]
		\label{#2}
		{#3}
	\end{myremark}
}

\newcommand{\mkConvention}[3]{
	\begin{myconv}[{#1}]
		\label{#2}
		{#3}
	\end{myconv}
}

\newcommand{\mkExercise}[3]{
	\begin{myexerc}[{#1}]
		\label{#2}
		{#3}
	\end{myexerc}
}
